\chapter{Einleitung} \label{chp:introduction}

\section{Projektbeschreibung} \label{sec:purpose}

Dieses Forschungsprojekt ist der erste Teil einer zweiteiligen Reihe über das Konzipieren und Entwickeln einer lokal verfügbaren \gls{restapi}. Dabei steht die Simulation eines oder mehrerer Serverendpunkte zur Abfrage von normalerweise entfernten Ressourcen im Mittelpunkt. Die Realisierung dieser Projekte erfordert die Auswahl geeigneter Technologien, die Analyse der Anforderungen sowie das Design und die Entwicklung der Software. Diese Aufgaben werden Teil des ersten Forschungsprojektes, während sich der zweite Teil der Entwicklung einer beispielhaften Software widmen könnte, welche an die im ersten Teil entwickelte \gls{restapi} angebunden wird. Das Gesamtpaket bildet schließlich eine vollständig aufeinander aufbauende Client-Server Lösung, welche ohne eine aktive Anbindung an das Internet genutzt werden kann. \\
\\ 
\section{Motivation} \label{sec:motivation}

Ein Projekt wie dieses wird zum Beispiel bei einer Präsentation benötigt, in welcher eine unzureichende Internetverbindung zu erwarten ist. So können Beispieldaten in einem Projekt wiedergegeben werden ohne auf einen entfernten Server zuzugreifen. 
Desweiteren ist in nahezu allen mittleren bis großen Unternehmen eine restriktive Internetnutzung standard. Das heißt, dass es während der Entwicklung von neuen Projekten nicht unbedingt möglich ist auf entfernte Ressourcen zuzugreifen. \cite{Hunt.1998}. Auch hier kann eine lokale \gls{restapi} sinnvoll sein, um mit Beispieldaten arbeiten zu können. 

\section{Vorgehensweise} \label{sec:approach}

Bei der Entwicklung der Software wird sich an dem Buch \textit{Software Engineering} von \textit{Ian Sommerville} orientiert \cite{Sommerville.2016}. Auf die Anforderungsanalyse wird dabei im \autoref{chp:design} eingegangen, welches ebenfalls die System- und Architekturmodellierung umfasst. Diese Ausarbeitungen werden dann im \autoref{chp:implementation} in die Praxis umgesetzt. Hier wird nun ein vollständiges Programm entwickelt, welches sowohl die \gls{api} als auch eine Benutzeroberfläche zur Anpassung der Endpunkte zur Verfügung stellt. Nach erfolgter Implementierung stellt das \autoref{chp:handbook} ein Handbuch dar, welches die Nutzung der Oberfläche und der \gls{api} illustriert. Im letzten Kapitel wird die Umsetzung überprüft und mit den Anforderungen verglichen. Ein Ausblickt gibt Aufschluss über mögliche Verbesserungen oder Erweiterungen. 